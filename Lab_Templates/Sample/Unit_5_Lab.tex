\documentclass[11pt]{article}
\usepackage[margin=1in]{geometry}
\usepackage[pdftex]{graphicx}
\usepackage{amsmath,amssymb,amsthm}
\usepackage{william} 
\usepackage{subcaption}
\usepackage{circuitikz} % for circuits
\usepackage{graphicx} % pictures
\usepackage{wrapfig} % for wrapping pictures
\usepackage{pgfplots} % graphs
\usepackage[makeroom]{cancel} % for striking out math
\usepackage{parskip} % consistent indentation
\usepackage{lettrine}
\usepackage{Zallman}

% headers and footers
\usepackage{fancyhdr}
\pagestyle{fancy}
\lhead{Will Lancer and James Cross}
\chead{}
\rhead{PHY 335 Unit 5 Lab Report}
\lfoot{}
\cfoot{\thepage}
\rfoot{}
\renewcommand{\headrulewidth}{0.4pt}
\setlength{\headheight}{14pt}

\linespread{1.03} % give a little extra room
\setlength{\parindent}{0.2in} % reduce paragraph indent a bit

% This is the magic: controls the font of the <letter>
\renewcommand{\LettrineFontHook}{\Zallmanfamily}

% This controls the font of the <text> 
\renewcommand{\LettrineTextFont}{\Huge\scshape}

% Bipoles Specifications
\ctikzset{bipoles/thickness=1.2, label distance=1mm, voltage shift = 1}

\DeclareSIUnit{\belmilliwatt}{Bm}
\DeclareSIUnit{\dBm}{\deci\belmilliwatt}

%%%%%%%%%%%%%%%%%%%%%%%%%%%%%%%%%%%%%%%%%%%%%%

% Feedback from last labs: 
% Procedures and data collection:
	% Keep measuring the true values of components used.
% Figures:
	% Include axes and labels on pictues of oscilloscope.
	% Circuit diagrams:
		% Label components with values directly.
		% Denote ground explicitly.
	% Make xlim on plots comforatably accomodate data.
	% If data is very spread out, make secondary plots zoomed in on the 
	% interesting part of the data.
% Data analysis
	% Agreement to uncertainty means consistency with zero, so if 0 is within 
	% (expected) - (measured) \pm (error prop for -), then it agrees to within
	% uncertainty.
	% If something does not agree to within uncertainty, note by how many
	% standard deviations (sigmas)

%%%%%%%%%%%%%%%%%%%%%%%%%%%%%%%%%%%%%%%%%%%%%%

\begin{document}

\thispagestyle{empty}

	\begin{center}
		\includegraphics[width = 0.4\textwidth]{Stony_Brook_U_logo_vertical.svg.png}
	\end{center}
	\vspace{5mm}
	\begin{center}
		\textbf{\begin{LARGE}
		PHY 335 Unit 5 Lab Report
		\end{LARGE}}
		\vspace{5mm}
	\end{center}
	\begin{center}
		{\large Prepared for the gaze of Gannon Lawley}\\
		\vspace{20mm}
	\end{center}
	\begin{center}
		\textbf{\large James Cross and Will Lancer}\\
		\vspace{20mm}
	\end{center}
	\begin{center}
	\includegraphics[width=.3\textwidth]{jamer.jpg}\hfill
	\includegraphics[width=.3\textwidth]{mee.jpg}\hfill
	\includegraphics[width=.3\textwidth]{gannon.jpg}
	\end{center}
	\vspace{0.7in}
	\begin{center}
		\textbf{\large Department of Physics}\\
		{\large State University of New York at Stony Brook}\\
		{\large November 2024\\}
	\end{center}

\newpage

	\begin{center}
	{\huge \textbf{PHY 335: Unit 5 Lab Report}}

	\vspace{5pt}

	{\textbf{Will Lancer and James Cross}}

	\vspace{5pt}

	{\textbf{November 4, 2024}}
	\vspace{5pt}

	{\Large \textbf{Abstract}}
	\end{center}
	\begin{raggedleft}
	\lettrine{T}{his} lab utilizes the operational amplifier (opamp) to produce 
	circuits not possible with passive components alone. In this lab, we 
	construct the voltage follower, non-inverting and inverting amplifiers, and 
	opamp differentiator. Additionally, we measure non-ideal properties of the
	opamp such as the slew rate.
	\end{raggedleft}

\newpage

% Introduction
\section{Introduction}

So, throughout this course, we've learned that basically
everything is a voltage divider; you put some voltage in,
and depending on the neat arrangement of impedances on your
breadboard, you get some (lower) voltage out. What if we
wanted \emph{more} voltage instead? In comes the veritable
\vocab{opamp}. This is the dream of experimentalists everywhere:
a linear circuit element with incredibly high input impedance,
incredibly low output inpedance, and very high/low gain.

Let's now get into some subtleties of opamps. The first is
their modus operandi: what are they ``trying to do''? The
one and only goal in the life of an opamp is to satisfy
the equation
\begin{align*}
	V_{\rm out} = A (V_+ - V_-).
\end{align*}
If you just run a voltage through the opamp, you can
get its \vocab{open loop gain}; this is just the
gain you get when the output isn't connected to the
input in any way. Theoretically, this gain is infinite;
in practice, it is just very high, usually something like $\sim 10^5$.
We can express things like this in math via the
\vocab{transfer function},
\begin{align*}
	G(\omega) = \frac{V_{\rm out}}{V_{\rm in}} = \frac{G_{\rm open}(\omega)}{1 + \beta G_{\rm open}(\omega)}
\end{align*}
Similar to how we do theory (make everything into a simple
harmonic oscillator), we would like to implement some stability
into the opamp, or in other words, implement a \vocab{negative feedback} loop.
In practice this is done by feeding some portion of $V_{\rm out}$
back into the inverting connection, $V_-$. When the opamp
is in the negative feedback regieme, the \vocab{Golden Rules}
apply; these are the statements
\begin{align*}
	& V_+ = V_-
	& I_+ = I_- = 0.
\end{align*}
These rules tell you how to analyze opamps theoretically.
We may apply these ideas to make some basic circuits, which
are all exactly what they sound like.
\begin{itemize}
	\item \vocab{Follower}. The voltage follower is just a circuit
	whose output is connected to its inverting input; by the Golden
	Rules, the output voltage tries to equal whatever $V_+$ is at.
	\item \vocab{Amplifiers}. There are three kinds of amplifiers:
	\vocab{inverting}, \vocab{non-inverting}, and \vocab{differential}.
	These do exactly what you think they do. The inverting amplifier
	produces a larger (or smaller), inverted signal. The non-inverting amplifier
	produces a larger, non-inverted signal, and the differential
	amplifer produces an output voltage proportional to the difference
	in input voltages for $V_+$ and $V_-$. The equations for these found
	via the Golden Rules are respectively $G = (R_1 + R_2)/R_2$, $G = -R_2/R_1$,
	and $V_{\rm out} = (R_2/R_1)(V_{in, +} - V_{\rm in, -})$. \label{amplifiers}
	\item \vocab{Current source}. If one wants a current source
	in a circuit, they can use an opamp to do so. Apply the Golden
	Rules to a circuit with a load connected to its output,
	and a resistor through its path to ground, and you will
	get a (voltage controlled) current source. The equation
	for this circuit is $I_{\rm load} = V_{\rm in}/R$. 
	\item \vocab{Differentiators and integrators}. If you simply
	attach a high and low pass circuit respectively through the
	output of an opamp, you will get a differentiator and integrator
	respectively. It's really that simple. The equations for these
	guys can be found via the Golden Rules, and are respectively
	$V_{\rm out} = - RC dV_{\rm in}/dt$ and 
	$V_{\rm out} = - (1/RC) \int V_{\rm in} + C$.
\end{itemize}
What's the update time of an opamp? The approximate answer
is ``really fast'', but what the answer isn't is ``instant'';
in comes the idea of \vocab{slew rate}. The slew rate is the
maximum rate at which an opamp can change its output voltage.
This is a frequency dependent and input voltage dependent idea,
as can be encapulated in the equation
\begin{align}
	SR = 2\pi V_0 f.
	\label{slew}
\end{align}

% Data and procedures
\newpage
\section{Data and procedures}

\subsection{Opamp logistics}

\noin
Looking at the data sheet \href{https://www.ti.com/lit/ds/symlink/tl081.pdf}{here}, 
we see that
\begin{itemize}
	\item \textbf{Maximum supply voltage:} $\pm 18\si{\volt}$ (or
	$36\si{\volt}$ across the terminals).
	\item \textbf{Open-loop amplification:} typically around $125\si{\decibel}$
	\item \textbf{Minimum load resistance:} the minimum load resistance
	for an output voltage $<18 \si{\volt}$ over a $0.25 \si{\watt}$ resistor is
	$\sim 440 \si{\ohm}$, based on figure 5-12 of the data sheet. 
\end{itemize}

\subsection{Voltage follower}

\subsubsection{DC voltage follower test}
We build a voltage divider controlled by a potentiometer and attach the output 
to the input of an opamp voltage follower as shown below:

\begin{figure}[H]
	\centering
	\begin{circuitikz}
%		%Grid
%		\def\length{4}
%		\draw[thin, dotted] (-\length,-\length) grid (\length,\length);
%		\foreach \i in {1,...,\length}
%		{
%			\node at (\i,-2ex) {\i};
%			\node at (-\i,-2ex) {-\i};	
%		}
%		\foreach \i in {1,...,\length}
%		{
%			\node at (-2ex,\i) {\i};	
%			\node at (-2ex,-\i) {-\i};	
%		}
%		\node at (-2ex,-2ex) {0};
	
		%Circuit
		% \node[op amp, noinv input up] at (0,0) (opamp) {};
		% \node[ground] at (-3.69,-3) (ground) {};
		% \draw[-latex] (opamp.up) -- ++(0,0.5) node [above] {$+15\si{\volt}$};
		% \draw[-latex] (opamp.down) -- ++(0,-0.5) node [below] {$-15\si{\volt}$};
		% \draw (opamp.out) to[short,-*] ++(1.5,0) node[shift={(0.6,0)}] {$V_{\rm out}$};
		% \draw (1.66,0) to[short,*-] ++(0,-2.5) -- ++(-3.5,0) -- ++(0,2) -- (opamp.-);
		% \draw (opamp.+) -- ++(-2.5,0) to[sV, l_=$V_{\rm in}$] (ground);
		\node[op amp, noinv input up] at (0,0) (opamp) {};
		\draw (opamp.up) to (-0.1,3)
		to (-6,3) to[battery1, l={$15 \si{\volt}$}] (-6,0)
		to (-6.5,0) node[ground]{};
		\draw (opamp.down) to (-0.1,-3)
		to (-6,-3) to[battery1, l_={$-15 \si{\volt}$}] (-6,0);
		\draw (opamp.-) to (-1.5, -0.5) 
		to (-1.5, -1.25) to (-1.25, -1.25) to [crossing] (1.05, -1.25) 
		to (1.2, -1.25) to (opamp.out) 
		to [rmeter,t={$V_\text{out}$}] (3,0) to node[ground]{} (3,0);
		\draw (-3,3) to [R,a={$10 \si{\kilo\ohm}$}] (-3,1) 
		to [american potentiometer,a={$10 \si{\kilo\ohm}$}, n=pot] (-3,-1) 
		to [R,a={$10 \si{\kilo\ohm}$}] (-3,-3);
		\draw (opamp.+) to (-1.5,0.5) to (pot.wiper);
		\draw (-1.5,0.5) to (-1.5,1.25) to (-1.25,1.25) 
		to [crossing] (1.05, 1.25) to (1.2, 1.25)
		to [rmeter,t={$V_\text{in}$}] (3,1.25) to node[ground]{} (3,1.25);
		
		
	\end{circuitikz}
	\caption{The voltage follower for the experiment.}
	\label{circ:follower}
\end{figure}


To test that the voltage follower matches the input and output voltages, we
sweep the input voltage and measure the input and output of the opamp.
If the voltage follower is working, then we expect $V_{\rm out} = V_{\rm in}$ 
within uncertainty.

\begin{table}[H]
	\centering
	\begin{tabular}{|l|l|l|l|}
	\hline
	$V_{\rm in}$ (V)   & $V_{\rm out}$ (V)  & $V_{\rm in} - V_{\rm out}$ (V) & Agreement within uncertainty? \\
	\hline
	$-4.907 \pm 0.001$ & $-4.904 \pm 0.002$ & $-0.003 \pm 0.002$           & No; $1.5$ sigma                  \\
	$-4.015 \pm 0.003$ & $-4.012 \pm 0.001$ & $-0.003 \pm 0.003$           & Yes                           \\
	$-3.023 \pm 0.002$ & $3.02 \pm 0.002$   & $-0.003 \pm 0.003$           & Yes                           \\
	$-1.914 \pm 0.001$ & $-1.912 \pm 0.001$ & $-0.002 \pm 0.001$           & No; $2$ sigma                 \\
	$-1.051 \pm 0.001$ & $-1.05 \pm 0.002$  & $-0.001 \pm 0.002$           & Yes                           \\
	$-0.022 \pm 0.002$ & $-0.021 \pm 0.001$ & $-0.001 \pm 0.002$           & Yes                           \\
	$1.036 \pm 0.003$  & $1.046 \pm 0.001$  & $-0.01 \pm 0.003$            & Yes                           \\
	$2.072 \pm 0.002$  & $2.081 \pm 0.001$  & $-0.009 \pm 0.002$           & No; $4.5$ sigma                 \\
	$3.081 \pm 0.002$  & $3.09 \pm 0.002$   & $-0.009 \pm 0.003$           & No; $3$ sigma                 \\
	$4.043 \pm 0.002$  & $4.05 \pm 0.002$   & $-0.007 \pm 0.003$           & No; $2.3$ sigma                 \\
	$4.951 \pm 0.002$  & $4.958 \pm 0.001$  & $-0.007 \pm 0.002$           & No; $3.5$ sigma                 \\
	\hline
	\end{tabular}
	\caption{A sweep of input voltages and the corresponding measured
	output voltages, as well as the agreement to within uncertainty
	for each measurement. We state the $\sigma$ difference if there
	is not agreement to within uncertainty.}
\end{table}

We can see that out of the $11$ measurements, $5$ of them
agreed to within uncertainty. We now consider the voltage divider.

\subsubsection{Loading the opamp}
To compare the voltage divider with the opamp follower with the divider without
an opamp, we put a load on the output. Setting the voltage to 
$\sim 5 \si{\volt}$, we calculate a Thevenin resistance of 
$R_{\rm Th} \approx 6600\si{\ohm}$ for the divider. We then choose a load 
resistor of $1 \si{\kilo\ohm}$ since it is both less than the Thevenin 
equivalent but larger than the minimum load impedance of 
$\approx 440 \si{\volt}$.
To test the divider with the voltage follower, attach a load resistor to the 
output of the opamp and measure the voltage. 
Remove the opamp, attach the load resistor to the output of the voltage 
divider, and measure the output voltage again.
\begin{table}[H]
	\centering
	\begin{tabular}{|c|c|c|}
		\hline
		 & $V_{\rm in}$ (V) & $V_{\rm out}$ (V)\\
		\hline
		Loaded & $4.934 \pm 0.001$ & $4.933 \pm 0.001$\\
		Unloaded & $4.934 \pm 0.001$ & $0.6497 \pm 0.003$\\
		\hline
	\end{tabular}
	\caption{Measured $V_{\rm in}$ and $V_{\rm out}$ values
	for the voltage follower.}
\end{table}

% measured resistor value
% calculate thevenin equivalent of divider based on choice of 10k and 20k ohm
% resistors
% note off zero of DMM for different days

\noin
\textbf{Remark:} Note that on the day of the experiment, all voltages were offset
by $0.006 \pm 0.001 \si{\volt}$, as the DMM ground voltage was this
number instead of zero.
\label{voltage_offset}

\subsubsection{AC voltage follower test}

Using our opamp voltage follower, we now feed in a
signal from the signal generator and measure the output
voltage vs. the input voltage as a function of the frequency
of the signal generator. We use two different input amplitudes:
$1 \pm 0.02 \si{\volt}$ and $10 \pm 0.01 \si{\volt}$.
We record the input and output voltages, and conclude
if they agree to within uncertainty. Below is the table for the 
$19.95 \si{\decibel}$ target 
gain, while the full collection of measurements are in the appendix. 

\begin{table}[H]
	\centering
	\begin{tabular}{|c|c|c|}
	\hline
	Frequency (Hz) & Input Voltage (V) & Output Voltage (V)\\
	\hline
	$10$ & $2.02 \pm 0.02$ & $1.98 \pm 0.02$ \\
	$20$ & $2.10 \pm 0.02$ & $2.14 \pm 0.04$ \\
	$45$ & $2.12 \pm 0.02$ & $2.14 \pm 0.02$ \\
	$100$ & $2.12 \pm 0.02$ & $2.16 \pm 0.02$ \\
	$200$ & $2.10 \pm 0.02$ & $2.14 \pm 0.02$ \\
	$450$ & $2.12 \pm 0.02$ & $2.14 \pm 0.02$ \\
	$1\text{k}$ & $2.12 \pm 0.02$ & $2.14 \pm 0.02$ \\
	$2\text{k}$ & $2.12 \pm 0.02$ & $2.14 \pm 0.02$ \\
	$4.5\text{k}$ & $2.12 \pm 0.02$ & $2.12 \pm 0.02$ \\
	$10\text{k}$ & $2.12 \pm 0.02$ & $2.10 \pm 0.04$ \\
	$20\text{k}$ & $2.12 \pm 0.02$ & $2.12 \pm 0.02$ \\
	$45\text{k}$ & $2.12 \pm 0.02$ & $2.10 \pm 0.02$ \\
	$100\text{k}$ & $2.12 \pm 0.02$ & $2.12 \pm 0.02$ \\
	$200\text{k}$ & $2.12 \pm 0.04$ & $2.12 \pm 0.02$ \\
	$450\text{k}$ & $2.12 \pm 0.02$ & $2.14 \pm 0.02$ \\
	$1\text{M}$ & $2.12 \pm 0.02$ & $2.16 \pm 0.04$ \\
	$2\text{M}$ & $2.06 \pm 0.02$ & $2.46 \pm 0.02$ \\
	\hline
	\end{tabular}
	\caption{Voltage measurements across different frequencies for channels 1 and 2,
	for the $\sim 1\si{\volt}$ input signal.}
	\label{tab:voltage_measurements}
\end{table}

\begin{table}[H]
	\centering
	\begin{tabular}{|c|c|c|}
	\hline
	Frequency (Hz) & Input Voltage (V) & Output Voltage (V) \\
	\hline
	$10$ & $20.0 \pm 0.02$ & $20.0 \pm 0.02$ \\
	$20$ & $20.0 \pm 0.02$ & $20.0 \pm 0.02$ \\
	$45$ & $20.0 \pm 0.02$ & $20.4 \pm 0.02$ \\
	$100$ & $20.4 \pm 0.02$ & $20.8 \pm 0.02$ \\
	$200$ & $20.0 \pm 0.02$ & $20.4 \pm 0.02$ \\
	$450$ & $20.0 \pm 0.02$ & $20.4 \pm 0.02$ \\
	$1\text{k}$ & $20.0 \pm 0.02$ & $20.4 \pm 0.02$ \\
	$2\text{k}$ & $20.0 \pm 0.02$ & $20.4 \pm 0.02$ \\
	$4.5\text{k}$ & $20.0 \pm 0.02$ & $20.4 \pm 0.02$ \\
	$10\text{k}$ & $20.0 \pm 0.02$ & $20.4 \pm 0.02$ \\
	$20\text{k}$ & $20.0 \pm 0.02$ & $20.4 \pm 0.02$ \\
	$45\text{k}$ & $20.0 \pm 0.02$ & $20.40 \pm 0.02$ \\
	$100\text{k}$ & $20.0 \pm 0.02$ & $20.4 \pm 0.02$ \\
	$200\text{k}$ & $20.0 \pm 0.04$ & $20.4 \pm 0.02$ \\
	$450\text{k}$ & $20.0 \pm 0.04$ & $16.4 \pm 0.02$ \\
	$1\text{M}$ & $20.0 \pm 0.02$ & $8.00 \pm 0.02$ \\
	$2\text{M}$ & $20.0 \pm 0.02$ & $5.20 \pm 0.4$ \\
	\hline
	\end{tabular}
	\caption{Voltage measurements across different frequencies for channels 1 and 2,
	for the $\sim 12\si{\volt}$ frequency.}
	\label{tab:voltage_measurements_chung}
\end{table}

\subsection{Non-inverting and inverting opamp}
% mention input voltage
% describe binary search-esque method for finding breakdown


\subsubsection{Non-inverting amplifier}

First we check that the $\pm 15 \si{\volt}$ knobs are
turned to the right. Then we connect them to the rails
of the circuit ($V_+ > 15 \si{\volt}$ and $V_- < - 15 \si{\volt}$). 
We then build the following circuit,
which will be our non-inverting amplifier:

\begin{figure}[H]
	\centering
	\begin{circuitikz}

		%		%Grid
		%		\def\length{4}
		%		\draw[thin, dotted] (-\length,-\length) grid (\length,\length);
		%		\foreach \i in {1,...,\length}
		%		{
		%			\node at (\i,-2ex) {\i};
		%			\node at (-\i,-2ex) {-\i};	
		%		}
		%		\foreach \i in {1,...,\length}
		%		{
		%			\node at (-2ex,\i) {\i};	
		%			\node at (-2ex,-\i) {-\i};	
		%		}
		%		\node at (-2ex,-2ex) {0};
				
			%Circuit
			\node[op amp, noinv input up] at (0,0) (opamp) {};
			\node[ground] at (-3.19,-2.5) {};
			\draw (-3.19,-1.65) to[pR,n=pr,a={$10\si{\kilo\ohm}$}] (-0.2,-1.65) 
			to (2.2,-1.65);
			\draw (opamp.-) -- ++(-0.5,0) -- (pr.wiper);
			\draw (opamp.+) -- ++(-2,0) to[sV, l_=$V_{\rm in}$] ++(0,-2) -- ++(0,-1);
			\draw (opamp.up) -- ++(0,0.5) node[right] {$V_+$};
			\draw (opamp.down) -- ++(0,-0.5) node[right] {$V_-$};
			\draw (opamp.out) to ++(1,0) 
			to ++(0,-1.65);
			\draw (opamp.out) to ++(2,0) to [rmeter,t={$V_\text{out}$}]
			++(0,-1.65) to node[ground]{} ++(0,0);
		
		\end{circuitikz}
		\caption{The non-inverting amplifier.}
\end{figure}

To test the non-inverting amplifier, we measure various gains
and various frequencies with gains between $0 \si{\decibel}$ and 
$26 \si{\decibel}$; specifically, we measure
a sweep of frequencies for each given gain value. We employ a psuedo
``binary-search'' algorithm, intended on finding the lowest frequency
that produces noticable differences in the predicted and measured gain,
i.e. when the slew rate has set in. The algorithm is implemented by:
\begin{enumerate}
	\label{binary_search}
	\item Measure the gain at $10\si{\hertz}$ and at $2\si{\mega \hertz}$. 
	Note that the slew rate will be salient for $2\si{\mega \hertz}$.
	\item Split the \emph{logarithmic} difference between $10\si{\hertz}$
	and $2\si{\mega \hertz}$; this comes out to be $1.414\si{\kilo\hertz}$.
	Measure the gain at this frequency.
	\item Measure the gain at this frequency; if the slew rate is present
	(the measured gain is lower than the predicted gain), go lower
	in frequency by splitting the logarithmic difference between
	the current frequency and $10\si{\hertz}$. If the slew rate is 
	not present go higher in frequency by splitting the logarithmic 
	difference between the current frequency and $2\si{\mega\hertz}$.
	\item Repeat step 3 until a frequency is found where the slew
	rate is present, i.e. a frequency such that the measured gain
	is less than the predicted gain.
\end{enumerate}
The results of this algorithm are collated into the tables
below:

\begin{table}[H]
	\centering
	\begin{tabular}{|c|c|c|c|c|}
	\hline
	Frequency (Hz)    & $V_{\rm in}$ (mV) & $V_{\rm out}$ (V) & Measured gain (dB) & Target gain (dB) \\
	\hline
	$10$              & $540 \pm 40$     & $10.4 \pm 0.2$    & $19.259 \pm 1.474$ & 19.95            \\
	$1.414\text{k}$   & $468 \pm 4$      & $10.1 \pm 0.1$    & $21.581 \pm 0.282$ & 19.95            \\
	$53.178\text{k}$  & $508 \pm 4$      & $10.2 \pm 0.2$    & $20.079 \pm 0.424$ & 19.95            \\
	$326.122\text{k}$ & $508 \pm 2$      & $4.48 \pm 0.08$   & $8.819 \pm 0.161$  & 19.95            \\
	$2\text{M}$       & $480 \pm 20$     & $1.20 \pm 0.02$   & $2.5 \pm 0.112$    & 19.95            \\
	\hline
	\end{tabular}
	\caption{Measured input voltage vs. output voltage at a range of
	frequencies, up until the slew rate is seen.}
\end{table}

\newpage
\subsubsection{Inverting amplifier}

First we check that the $\pm 15 \si{\volt}$ were
turned to the right. Then we connect them to the rails
of the circuit. We then built the following circuit,
which will be our inverting amplifier:
\begin{figure}[H]
	\centering
	\begin{circuitikz}

		%		%Grid
		%		\def\length{4}
		%		\draw[thin, dotted] (-\length,-\length) grid (\length,\length);
		%		\foreach \i in {1,...,\length}
		%		{
		%			\node at (\i,-2ex) {\i};
		%			\node at (-\i,-2ex) {-\i};	
		%		}
		%		\foreach \i in {1,...,\length}
		%		{
		%			\node at (-2ex,\i) {\i};	
		%			\node at (-2ex,-\i) {-\i};	
		%		}
		%		\node at (-2ex,-2ex) {0};
				
		% %Circuit
		% \node[op amp] at (0,0) (opamp) {};
		% \node[ground] at (-3.19,-2.5) {};
		% \draw (opamp.+) -- ++(-0.5,0) -- ++(0,-1.15) to[short,-*] ++(-1.5,0);
		% \draw (opamp.-) -- ++(-2,0) to[sV, l_=$V_{\rm in}$] ++(0,-2) -- ++(0,-1);
		% \draw[-latex] (opamp.up) -- ++(0,1) node[above] {$V_+$};
		% \draw[-latex] (opamp.down) -- ++(0,-1) node[below] {$V_-$};
		% \draw (opamp.out) to[short, -*] ++(1,0) node[shift={(0.6,0)}] {$V_{\rm out}$};
	
		\node[op amp, noinv input up] at (0,0) (opamp) {};
		\draw (-3.19,-1.65) to[pR,n=pr,a={$10\si{\kilo\ohm}$}] (-0.2,-1.65) 
		to (2.2,-1.65);
		\draw (-3.19,-1.65) to[sV, l_=$V_{\rm in}$] ++(0,-1) node[ground]{};
		\draw (opamp.-) -- ++(-0.5,0) -- (pr.wiper);
		\draw (opamp.+) -- ++(-1,0) node[ground]{}; %to[sV, l_=$V_{\rm in}$] ++(0,-2) -- ++(0,-1);
		\draw (opamp.up) -- ++(0,0.5) node[right] {$V_+$};
		\draw (opamp.down) -- ++(0,-0.5) node[right] {$V_-$};
		\draw (opamp.out) to ++(1,0) 
		to ++(0,-1.65);
		\draw (opamp.out) to ++(2,0) to [rmeter,t={$V_\text{out}$}]
		++(0,-1.65) to node[ground]{} ++(0,0);

		\end{circuitikz}
		\caption{The inverting amplifier.}
\end{figure}

We test this circuit via the algorithm (\ref{binary_search}).
We use the oscilloscope to display the output voltages and
measure the gain.
Below is the data for the $19.953 \si{\decibel}$ target gain measurements.
The rest of the data can be found in the appendix.

\begin{table}[H]
	\centering
	\begin{tabular}{|c|c|c|c|c|}
	\hline
	Frequency (Hz)    & $V_{\rm in}$ (mV) & $V_{\rm out}$ (V) & Measured gain (dB) & Target gain (dB) \\
	\hline
	$10$              & $460 \pm 4$      & $9.76 \pm 0.08$   & $21.217 \pm 0.254$ & 19.953           \\
	$1.414\text{k}$   & $448 \pm 4$      & $9.76 \pm 0.08$   & $21.786 \pm 0.264$ & 19.953           \\
	$53.178\text{k}$  & $440 \pm 4$      & $9.12 \pm 0.08$   & $20.727 \pm 0.262$ & 19.953           \\
	$326.122\text{k}$ & $480 \pm 4$      & $3.84 \pm 0.08$   & $8.0 \pm 0.18$     & 19.953           \\
	$2\text{M}$       & $480 \pm 8$      & $.712 \pm 0.002$  & $1.483 \pm 0.025$  & 19.953           \\
	\hline
	\end{tabular}
	\caption{Measured input voltage vs. output voltage at a range of
	frequencies, up until the slew rate is seen for the inverting amplifier.}
\end{table}

\newpage
\subsection{Differentiator}

First, we tune the $+15 \si{\volt}$ and $-15 \si{\volt}$ output of the DC power 
supply to $V_{\text{CC}+} = (15.049 \pm 0.10) \si{\volt}$ 
(uncertainty estimated by variations in the value) and
$V_{\text{CC}-} = -(15.014 \pm 0.005) \si{\volt}$.
Then, we connect them to the rail voltages in the opamp.
Next, we build the differentiator circuit pictured below and measure the output
with the oscilloscope $V_\text{out}$.
Finally, we measure the true value of the circuit components:
$R = (98.51 \pm 0.01) \si{\kilo\ohm}$ and $C =(94.7 \pm 0.01) \si{\nano\farad}$.

\begin{figure}[H]
	\centering
	\begin{circuitikz}
		\draw
			(0,0) node[op amp] (opamp) {}
			(-3,-1.5) to node[pos=1,left]{b} (-3,-1)
			to [sV={$V_\text{in}$}] (-3,1)
			to node[pos=0,left]{r} (-3,1.5)
			to [C={$C$}, a={$100 \si{\nano\farad}$}] (-1.2, 1.5)
			to  (opamp.-)
			(-1.2, 1.5) to [R={$R$}, a={$100 \si{\kilo\ohm}$}] (1.2,1.5)
			to (opamp.out)
			(-3, -1.5) to (-1.2, -1.5)
			to (opamp.+)
			(-1.2, -1.5) to node[ground]{} (-1.2, -2) 
			(-1.2, -1.5) to node[pos=1,below]{b} (-0.7, -1.5) 
			to [rmeter, t={$V_\text{out}$}] (1.2,-1.5) 
			to node[pos=0,right]{r} (opamp.out)
			;
	\end{circuitikz}
	\caption{
		Circuit diagram for differentiating circuit. The circuit component 
		values shown approximate the measured values of the components used. 
		Measured values are $R = (98.51 \pm 0.01) \si{\kilo\ohm}$ and 
		$C =(94.7 \pm 0.01) \si{\nano\farad}$.
	}
	\label{circ:diff}
\end{figure}

We tested the opamp with the following combinations of inputs and measured the
$V_\text{pp,out}$. Because of the instability of the output waveforms, the
oscilloscope sometimes mismeasured the $V_\text{pp}$. When this is the case, the 
$V_\text{pp}$ is estimated from the picture in tables 
\ref{tab:diff_output_triangle}-\ref{tab:diff_output_sine}. Manually measured
$V_\text{pp,out}$'s are marked with an asterisk (*). 
\begin{table}[H]
	\centering
	\begin{tabular}{|c|c|c|c|c|}
		\hline
		Number 	& Waveform 		& $V_{\text{pp,in}}$ 							& $V_{\text{pp,out}}$ 				& Frequency  \\
		\hline                                                                                                                  
		1		& Triangle 		& $(256 \pm 60) \si{\milli\volt}$				& *$(18.2 \pm 0.2) \si{\volt} $		& $1 \si{\kilo\hertz}$ \\
		\hline
		2		& Triangle 		& $(272 \pm 60) \si{\milli\volt}$				& *$(1.0 \pm 0.2 \si{\volt}$		& $100 \si{\hertz}$ \\
		\hline
		3		& Triangle 		& $(272 \pm 60) \si{\milli\volt}$				& *$(400 \pm 100 \si{\milli\volt}$	& $50 \si{\hertz}$ \\
		\hline
		4		& Square		& $(388 \pm 60) \si{\milli\volt}$				& $(29.0 \pm 0.2) \si{\volt}$		& $1 \si{\kilo\hertz}$ \\
		\hline
		5		& Square 		& $(380 \pm 60) \si{\milli\volt}$				& $(29.2 \pm 0.2) \si{\volt}$		& $100 \si{\hertz}$ \\
		\hline
		6		& Square		& $(372 \pm 60) \si{\milli\volt}$ 				& $(29.0 \pm 0.2)\si{\volt}$		& $50 \si{\hertz}$ \\
		\hline
		7		& Sine			& $(276 \pm 60) \si{\milli\volt}$				& $(12.8 \pm 0.2) \si{\volt}$		& $1 \si{\kilo\hertz}$ \\
		\hline
		8		& Sine			& $(280 \pm 60) \si{\milli\volt}$				& *$(1.6 \pm 0.6) \si{\volt}$		& $100 \si{\hertz}$ \\
		\hline
		9		& Sine			& $(276 \pm 60) \si{\milli\volt}$				& *$(800 \pm 200) \si{\milli\volt}$	& $50 \si{\hertz}$ \\
		\hline
	\end{tabular}
	\caption{
		Table of inputs and $V_\text{pp,out}$ tested for the differentiator 
		circuit in Figure \ref{circ:diff}. $V_\text{pp,out}$ was measured
		either from the oscilloscope (unmarked) or manually (marked with *).
		For manual measurements error is the larger between the size of the 
		smallest subdivision and the peak thickness of the line.
	}
	\label{tab:diff_inputs}	
\end{table}

Using the same numbering, the oscilloscope gave the following outputs.
As a sample of the full table of outputs provided in tables 
\ref{tab:diff_output_triangle}-\ref{tab:diff_output_sine}
in the appendix, below are the outputs of number 2 and 5.

\begin{minipage}{0.45\textwidth}
	\begin{figure}[H]
		\centering
		\includegraphics[width=\textwidth]{./Figures/diff_2.jpg}
		\caption{
			Pictured above is the oscilloscope output for number 2 in tables
			\ref{tab:diff_inputs} and 
			\ref{tab:diff_output_triangle}-\ref{tab:diff_output_sine}.
			This corresponded
			to the triangle wave input at $V_\text{pp}=(272 \pm 60) \si{\volt}$ 
			and $f=200 \si{\hertz}$. 
			The input is in yellow with 
			$100 \si{\milli\volt}/\text{div}$ and 
			$2 \si{\milli\second}/\text{div}$, and 
			the output is in blue with 
			$2.00 \si{\volt}/\text{div}$ and 
			$2 \si{\milli\second}/\text{div}$. 
		}
	\end{figure}
\end{minipage}
\hspace{0.1\textwidth}
\begin{minipage}{0.45\textwidth}
	\begin{figure}[H]
		\centering
		\includegraphics[width=\textwidth]{./Figures/diff_5.jpg}
		\caption{
			Pictured above is the oscilloscope output for number 5 in tables
			\ref{tab:diff_inputs} and
			\ref{tab:diff_output_triangle}-\ref{tab:diff_output_sine}.
			This corresponded
			to the square wave input at $V_\text{pp}=(380 \pm 60) \si{\volt}$ 
			and $f=200 \si{\hertz}$. 
			The input is in yellow with
			$100 \si{\milli\volt}/\text{div}$ and 
			$1 \si{\milli\second}/\text{div}$, and 
			the output is in blue with 
			$5.00 \si{\volt}/\text{div}$ and 
			$1 \si{\milli\second}/\text{div}$. 
		}
	\end{figure}
\end{minipage}

\subsection{Slew rate}

We measure the slew rate using the voltage follower circuit \ref{circ:follower}
with signal generator as $V_\text{in}$ in two separate ways. 

In the first method, we set the input to a sine wave and use the math function
to take the difference between the input and output of the voltage follower. 
Then increase the voltage until the difference function significantly deviates
from $0 \si{\volt}$. When take this voltage as the distortion voltage in
equation (\ref{slew}). The measured distortion frequency was measured to be
$(140 \pm 5) \si{\kilo\hertz}$, uncertainty based off of intervals between 
measurements.

\begin{minipage}{0.45\textwidth}
	\begin{figure}[H]
		\centering
		\includegraphics[width=\textwidth]{./Figures/dist_1.jpg}
		\caption{
			Picture of $V_\text{in}$ (yellow) and $V_\text{in} - V_\text{out}$ 
			(purple). While there is very high uncertainty due to line 
			thickness, $V_\text{in} - V_\text{out}$ has structure.
		}
	\end{figure}
\end{minipage}
\hspace{0.1\textwidth}
\begin{minipage}{0.45\textwidth}
	\begin{figure}[H]
		\centering
		\includegraphics[width=\textwidth]{./Figures/dist_2.jpg}
		\caption{
			Picture of $V_\text{in}$ (yellow) and $V_\text{in} - V_\text{out}$ 
			(purple). While there is very high uncertainty due to line 
			thickness, $V_\text{in} - V_\text{out}$ has structure.
		}
	\end{figure}
\end{minipage}

In the second method, we set the input to a square wave and measured points to
find the slope/slew rate. The points are 
$(-3.200 \pm 0.1) \si{\volt}$ at $(616.0 \pm 4.0) \si{\nano\second}$ and
$(4.300 \pm 0.1) \si{\volt}$ at $(948.0 \pm 4.0) \si{\nano\second}$.

\subsection{Trigger Circuit}
\label{sec:trigger_data}

First, we checked that the $+15 \si{\volt}$ and $-15 \si{\volt}$ knobs on the 
DC power supply were turned to the right.
Then, we used the DC power supply for the rail voltage and built the following 
circuit. 
We then tested the circuit by placing the grounded lead 
of the oscilloscope at the common grounds and the power lead at Output 1 before
moving it to Output 2.
Afterwards, we measured the component values. 

\begin{figure}[H]
	\centering
	\begin{circuitikz}
	\draw
		(0,0) node[op amp] (opamp) {}
		(opamp.+) to (-1.5,-0.5) 
		to node[ground]{} (-1.5,-1)
		(-1.2, 0.5) to (-1.2, 2)
		to [C={$C$}, a={$10 \si{\nano\farad}$}] (1.2, 2)
		to (1.2, 0)
		(opamp.-) to (-3,0.5)
		to [R={$R_1$}, a={$100 \si{\kilo\ohm}$}] (-3,-6)
		(opamp.out) to (1.2,0)
		to node[pos=1,right]{Output 1} (3,0)
		to (3,-1.5)
		to [R={$R_2$}, a={$2 \si{\kilo\ohm}$}] (-1.2, -1.5)
		to (-1.2, -2.5)
		;
	\draw
		(0,-4) node[op amp, yscale=-1] (opamp2) {}
		(-3,-6) to (1.2,-6) 
		to node[pos=1,right]{Output 2} (1.2, -4)
		to (opamp2.out)
		(opamp2.-) to (-1.5,-4.5) 
		to node[ground]{} (-1.5,-5)
		(opamp2.+) to (-1.2, -2.5)
		to [R={$R_3$}, a={$10 \si{\kilo\ohm}$}] (1.2,-2.5) 
		to (1.2,-6)
		;
	\end{circuitikz}
	\caption{Circuit diagram for triggering circuit. The circuit component 
		values shown approximate the measured values:
		$R_1 = (98.43 \pm 0.05) \si{\kilo\ohm} \approx 100 \si{\kilo\ohm}$, 
		$R_2 = (1.9597 \pm0.0005) \si{\kilo\ohm} \approx 2 \si{\kilo\ohm}$, 
		$R_3 = (9.825 \pm 0.005) \si{\kilo\ohm} \approx 10 \si{\kilo\ohm}$, and 
		$C = (10.2 \pm 0.02) \si{\nano\farad} \approx 10 \si{\nano\farad}$.
		The defining feature of the circuit is its utilization of both positive
		and negative feedback.
	}
\end{figure}

At Output 1 and Output 2 respectively, the voltage was a constant 
$(14.65 \pm 0.35) \si{\volt}$, and
$-(14.10 \pm 0.40) \si{\volt}$, when
judging uncertainty by line thickness.

\begin{figure}[H]
	\centering
	\includegraphics[width=\textwidth]{./Figures/scope_8_1.jpg}
	\caption{
		Pictured is the oscilloscope reading for the Output 1 at
		$5 \si{volt}/\text{div}$ and $200 \si{\micro\second}/\text{div}$. 
		The red marks the $0 \si{\volt}$ line. Output 1 appears as a thick
		constant line at $(14.65 \pm 0.35) \si{\volt}$.
	}
\end{figure}
\begin{figure}[H]
	\centering
	\includegraphics[width=\textwidth]{./Figures/scope_8_2.jpg}
	\caption{
		Pictured is the oscilloscope reading for the Output 2 at
		$5 \si{volt}/\text{div}$ and $200 \si{\micro\second}/\text{div}$. 
		The red marks the $0 \si{\volt}$ line. Output 2 appears as a thick
		constant line at $-(14.10 \pm 0.40) \si{\volt}$.
	}
\end{figure}


\newpage
% Analysis
\section{Analysis}

\subsection{Voltage follower}

% expectation is that all the values from the table agree to within 
% uncertainty of each other

\subsubsection{DC follower test}
For the voltage follower, we expect that all values 
from the measurement agree to within uncertainty; this is false. 
Only 5/11 of the measurements agree to within uncertainty. 
This may be due to faulty circuit connections, or inaccurate measuring 
equipment. There is also a systematic error with the DMM; the zero of 
the DMM was offset by $0.008\si{\volt}$ on the day of the experiment, 
as stated in remark (\ref{voltage_offset}).

\subsubsection{Load test}
Our expectations were
achieved; the divider and opamp follower's 
$V_{\rm in} = V_{\rm out}$ to within uncertainty, and the
voltage divider's $V_{\rm in}$ and $V_{\rm out}$ disagreed to within
uncertainty.

% Expect the Vin and Vout of the follower with the load agree but the Vin and
% Vout of the divider alone disagree

\subsubsection{AC follower test}
As we can see from the tables for the AC follower test,
the input and output voltage agree to within uncertainty
up to the slew rate cutoff frequency, at which they begin
to disagree. This is at $f = 2\si{\mega\hertz}$ for the $\sim 1\si{\volt}$
measurement, and at $f = 450\si{\kilo\hertz}$ for the $\sim 12\si{\volt}$ 
measurement. 

\subsection{Inverting and non-inverting amplifiers}

Note that remark (\ref{voltage_offset}) applies for the
non-inverting and inverting opamp experiments.
We plot the data in the graphs below:

\begin{figure}[H]
	\centering
	 \begin{minipage}{0.45\textwidth}
		 \centering
		 \includegraphics[width=0.9\textwidth]{Figures/exp4_graph.png} % first figure itself
		 \caption{Frequency vs. target gain from the non-inverting amplifier.}
	\end{minipage}\hfill
    \begin{minipage}{0.45\textwidth}
	    \centering
	    \includegraphics[width=0.9\textwidth]{Figures/exp5_graph.png} % second figure itself
		 \caption{Frequency vs. target gain from the inverting amplifier.}
    \end{minipage}
\end{figure}

The points in green are the ones where the predicted
and measured gain agree to within uncertainty; the points
below the green line are the points we predict to agree
with experiment within uncertainty. We have one outlier,
that is the $2$nd data point from the left in the bottom row.
For the inverting amplifier, as is evident from the data,
we saw the inversion. Possible explanations for the outlier
in the data is the DMM offset remark noted previously, or
faulty connections.

\subsection{Differentiator}
When comparing the output waveforms to the expected waveforms, the 
differentiator seems to qualitatively work better at lower frequencies for the 
triangle and square waves, but better at higher frequencies for the sine wave.

For the high frequency triangle and square wave measurements in particular, the
output waveform appears like it is trying to ``settle" into the correct 
waveform. Only the $2 \si{\kilo\hertz}$ measurements for both seem to not 
settle in finite time. This is unexpected, since the data sheet for the TL082
specifies a settling time of $0.1 \si{\micro\second}$ for better than $20 \%$
accuracy, but the opamp is failing to settle in $500 \si{\micro\second}$.

The expected $V_\text{pp,out}$ from (\ref{amplifiers}) 
seem to agree with measurement for the lower frequencies tested.

\begin{table}[H]
	\centering
	\begin{tabular}{|c|c|c|c|}
		\hline
		Number 	& $V_\text{pp,in}$ 					& Expected $V_\text{pp,out}$ 	& Measured $V_\text{pp,out}$ \\
		\hline
		1		& $(256 \pm 60) \si{\milli\volt}$	& $(9.55 \pm 2.24) \si{\volt}$	& $(18.2 \pm 0.2) \si{\volt}$ \\
		\hline
		2		& $(272 \pm 60) \si{\milli\volt}$	& $(1.01 \pm 0.22) \si{\volt}$	& $(1 \pm 0.2) \si{\volt}$ \\
		\hline
		3		& $(272 \pm 60) \si{\milli\volt}$	& $(507 \pm 112)\si{\milli\volt}$ & $(400 \pm 100) \si{\milli\volt}$ \\
		\hline
		4		& $(388 \pm 60) \si{\milli\volt}$	& N/A							& $(29.0 \pm 0.2) \si{\volt}$ \\
		\hline
		5		& $(380 \pm 60) \si{\milli\volt}$	& N/A							& $(29.2 \pm 0.2) \si{\volt}$ \\
		\hline
		6		& $(372 \pm 60) \si{\milli\volt}$	& N/A							& $(29.0 \pm 0.2) \si{\volt}$ \\
		\hline
		7		& $(276 \pm 60) \si{\milli\volt}$	& $(16.2 \pm 3.5) \si{\volt}$	& $(12.8 \pm 0.2) \si{\volt}$ \\
		\hline
		8		& $(280 \pm 60) \si{\milli\volt}$	& $(1.64 \pm 0.35) \si{\volt}$	& $(1.6 \pm 0.6) \si{\volt}$ \\
		\hline
		9		& $(276 \pm 60) \si{\milli\volt}$	& $(0.81 \pm 0.18) \si{\volt}$	& $(0.8 \pm 0.2) \si{\volt}$\\
		\hline
	\end{tabular}
	
\end{table}

\subsubsection{Error Analysis}
An explanation for why the quality waveform of the sine wave follows an 
opposite trend of the triangle and square waves is the presence of a ground 
loop in the system. This would also explain why there is so much uncertainty 
(thickness) in the oscilloscope input signals found in the appendix.

The disagreement at high frequencies seems to be due to the system not settling
into a stable output. This is unexpected as previously stated due to not 
meeting the specifications. Since this appears to be the opamp failing to meet
specification, this may be due to an error with the opamp itself.

Additionally, the input signal has an unusually large uncertainty. When
disconnected from the rest of the circuit, the signal quality significantly 
improved. This is due to the presence of a ground loop in the construction of 
the circuit causing many small, rapid oscillations in the signal.

\subsection{Slew rate}

In the distortion frequency method, the slew rate calculated from formula 
(\ref{slew}) is 
$(17.9 \pm 0.6) \si{\volt / \micro\second}$. 
In the slope method, the slew rate calculated is
$(22.0 \pm 0.6) \si{\volt / \micro\second}$.
When comparing to the $20 \si{\volt / \micro\second}$ in the datasheet, this 
makes the slew rate for the slope method better than specification, but 
worse for the distortion frequency method.

\subsubsection{Error Analysis}

The distortion frequency method may have underestimated the slew rate due to 
some vagueness in constitutes a ``distortion." With the given scales used, 
a distortion constitues a deviation by up to $1 \si{\volt}$ for a 
$V_\text{pp,in} = (20.4 \pm 0.4) \si{\volt}$ input signal. This may be too 
small of an error and would lead to an underestimate.

\subsection{Trigger Circuit}

It is as expected that the voltage is at a constant 
$+15 \si{\volt}$ at Output 1 and a constant $-15 \si{\volt}$ at Output 2. 
The constant behavior observed for both outputs in section 
\ref{sec:trigger_data} lines up with expectations. However, only the 
$(14.65 \pm 0.35) \si{\volt}$ measured at Output 1 agrees to within 
uncertainty with the expected $+15 \si{\volt}$. The measured Output 2 
$(14.10 \pm 0.40 \si{\volt}$ does not agree with the expected $-15 \si{\volt}$
by $2.25 \sigma$. This is a significant deviation.


\subsubsection{Error Analysis}
A systematic error in the procedure was not measuring the exact rail voltages
supplying the two opamps. Undervolting the $V_{\text{CC}-}$ rail could have 
produced a lower output at Output 2. By not measuring, we also lacked estimates 
on the uncertainties in the power supply and could not account for them.



% Conclusion
\section{Conclusion}

Using a TL082 opamp, we built a voltage follower.
The voltage follower matched theoretical predictions for
5/11 measurements; possible reasons that the other measurements
did not match predictions is because of faulty connections,
or because of the inaccurate zero of the DMM that day.

Attaching a load resistor to the output of our follower,
$V_{\rm out} = V_{\rm in}$ to within uncertainty, and
$V_{\rm out} \neq V_{\rm in}$ to within uncertainty
for the divider alone. 

Driving our voltage follower with an AC input signal,
we observed that the following behavior was accurate
up to the slew rate cutoff; this was at $\sim 450\si{\kilo\hertz}$
for the $\sim 1 \si{\volt}$ input, and at $2\si{\mega\hertz}$
for the $\sim 12 \si{\volt}$ signal.

We built non-inverting and inverting amplifiers, and swept
them through a range of frequencies and gains to observe their behaviors.
Theoretical predictions were largely matched; there was one outlier,
which may be explained through faulty connections. The
slew rate was observed for both amplifiers, at
identical frequencies and gains. Note that we saw the
inversion for the inverting amplifier.

We built a differentiator and swept it through a range
of frequencies; qualitatively, the differentiator worked
better for lower frequencies for the triangle and square
waves, and at higher frequencies for the sine waves.
A ground loop was present for the sine wave measurements,
explaining its large error and opposite behavior to the
triangle and square waves. The differentiator largely
did not match theoretical predictions, though.

We calculated the slew rate in two different ways,
and the numbers did not agree to within uncertainty;
both were off by around $2 \si{\volt}/\si{\micro\second}$
from the predicted $20\si{\volt}/\si{\micro\second}$.

We built a trigger circuit. The measured behavior matched
the theoretically predicted behavior for the constancy
measurements, but only the $\sim 15\si{\volt}$ measurement
at Output $1$ agrees to within uncertainty to its predicted value.

\newpage
\section{Appendix}

\subsection{Full non-inverting amplifier data}

\begin{table}[H]
	\centering
	\begin{tabular}{|l|l|l|l|l|}
	\hline
	Frequency (Hz)    & $V_{\rm in}$ (mV) & $V_{\rm out}$ (V) & Measured gain (dB) & Target gain (dB) \\
	\hline
	$10$              & $540 \pm 40$     & $10.4 \pm 0.2$    & $19.259 \pm 1.474$ & 19.95            \\
	$1.414\text{k}$   & $468 \pm 4$      & $10.1 \pm 0.1$    & $21.581 \pm 0.282$ & 19.95            \\
	$53.178\text{k}$  & $508 \pm 4$      & $10.2 \pm 0.2$    & $20.079 \pm 0.424$ & 19.95            \\
	$326.122\text{k}$ & $508 \pm 2$      & $4.48 \pm 0.08$   & $8.819 \pm 0.161$  & 19.95            \\
	$2\text{M}$       & $480 \pm 20$     & $1.20 \pm 0.02$   & $2.5 \pm 0.112$    & 19.95            \\
	\hline
	\end{tabular}
	\caption{Measured input voltage vs. output voltage at a range of
	frequencies, up until the slew rate is seen.}
\end{table}

\begin{table}[H]
	\centering
	\begin{tabular}{|l|l|l|l|l|}
	\hline
	Frequency (Hz)    & $V_{\rm in}$ (mV) & $V_{\rm out}$ (V) & Measured gain (dB) & Target gain (dB) \\
	\hline
	$10$              & $508 \pm 8$      & $5.44 \pm 0.04$   & $10.709 \pm 0.186$ & 10               \\
	$1.414\text{k}$   & $500 \pm 4$      & $5.20 \pm 0.04$   & $10.4 \pm 0.115$   & 10               \\
	$53.178\text{k}$  & $488 \pm 4$      & $5.04 \pm 0.04$   & $10.328 \pm 0.118$ & 10               \\
	$326.122\text{k}$ & $488 \pm 4$      & $4.20 \pm 0.08$   & $8.607 \pm 0.178$  & 10               \\
	$2\text{M}$       & $480 \pm 8$      & $0.8 \pm 0.04$    & $1.667 \pm 0.088$  & 10               \\
	\hline
	\end{tabular}
	\caption{Measured input voltage vs. output voltage at a range of
	frequencies, up until the slew rate is seen.}
\end{table}

\begin{table}[H]
	\centering
	\begin{tabular}{|l|l|l|l|l|}
	\hline
	Frequency (Hz)    & $V_{\rm in}$ (mV) & $V_{\rm out}$ (V) & Measured gain (dB) & Target gain (dB) \\
	\hline
	$10$              & $512 \pm 4$      & $2.46 \pm 0.04$   & $4.805 \pm 0.087$  & 5                \\
	$1.414\text{k}$   & $504 \pm 4$      & $2.50 \pm 0.02$   & $4.96 \pm 0.056$   & 5                \\
	$53.178\text{k}$  & $484 \pm 8$      & $2.48 \pm 0.02$   & $5.124 \pm 0.094$  & 5                \\
	$326.122\text{k}$ & $484 \pm 4$      & $2.50 \pm 0.02$   & $5.165 \pm 0.059$  & 5                \\
	$2\text{M}$       & $480 \pm 8$      & $0.84 \pm 0.02$   & $1.75 \pm 0.051$   & 5                \\
	\hline
	\end{tabular}
	\caption{Measured input voltage vs. output voltage at a range of
	frequencies, up until the slew rate is seen.}
\end{table}

\begin{table}[H]
	\centering
	\begin{tabular}{|l|l|l|l|l|}
	\hline
	Frequency (Hz)    & $V_{\rm in}$ (mV) & $V_{\rm out}$ (V) & Measured gain (dB) & Target gain (dB) \\
	\hline
	$10$              & $508 \pm 4$      & $1.60 \pm 0.02$   & $3.15 \pm 0.047$   & 3.16             \\
	$1.414\text{k}$   & $496 \pm 4$      & $1.62 \pm 0.02$   & $3.266 \pm 0.048$  & 3.16             \\
	$53.178\text{k}$  & $484 \pm 8$      & $1.60 \pm 0.02$   & $3.306 \pm 0.069$  & 3.16             \\
	$326.122\text{k}$ & $484 \pm 4$      & $1.58 \pm 0.02$   & $3.264 \pm 0.049$  & 3.16             \\
	$807.617\text{k}$ & $492 \pm 8$      & $1.52 \pm 0.04$   & $3.089 \pm 0.096$  & 3.16             \\
	$1.27\text{M}$    & $488 \pm 4$      & $1.20 \pm 0.02$   & $2.459 \pm 0.046$  & 3.16             \\
	$2\text{M}$       & $476 \pm 2$      & $0.900 \pm 0.02$  & $1.891 \pm 0.043$  & 3.16             \\
	\hline
	\end{tabular}
	\caption{Measured input voltage vs. output voltage at a range of
	frequencies, up until the slew rate is seen for the non-inverting
	amplifier.}
\end{table}

\subsection{Full inverting amplifier data}

\begin{table}[H]
	\centering
	\begin{tabular}{|l|l|l|l|l|}
	\hline
	Frequency (Hz)    & $V_{\rm in}$ (mV) & $V_{\rm out}$ (V) & Measured gain (dB) & Target gain (dB) \\
	\hline
	$10$              & $460 \pm 4$      & $9.76 \pm 0.08$   & $21.217 \pm 0.254$ & 19.953           \\
	$1.414\text{k}$   & $448 \pm 4$      & $9.76 \pm 0.08$   & $21.786 \pm 0.264$ & 19.953           \\
	$53.178\text{k}$  & $440 \pm 4$      & $9.12 \pm 0.08$   & $20.727 \pm 0.262$ & 19.953           \\
	$326.122\text{k}$ & $480 \pm 4$      & $3.84 \pm 0.08$   & $8.0 \pm 0.18$     & 19.953           \\
	$2\text{M}$       & $480 \pm 8$      & $.712 \pm 0.002$  & $1.483 \pm 0.025$  & 19.953           \\
	\hline
	\end{tabular}
	\caption{Measured input voltage vs. output voltage at a range of
	frequencies, up until the slew rate is seen for the inverting amplifier.}
\end{table}

\begin{table}[H]
	\centering
	\begin{tabular}{|l|l|l|l|l|}
	\hline
	Frequency (Hz)    & $V_{\rm in}$ (mV) & $V_{\rm out}$ (V) & Measured gain (dB) & Target gain (dB) \\
	\hline
	$10$              & $480 \pm 4$      & $5.20 \pm 0.02$   & $10.833 \pm 0.099$ & 10               \\
	$1.414\text{k}$   & $480 \pm 8$      & $5.20 \pm 0.02$   & $10.833 \pm 0.185$ & 10               \\
	$53.178\text{k}$  & $460 \pm 4$      & $5.2 \pm 0.02$    & $11.304 \pm 0.107$ & 10               \\
	$326.122\text{k}$ & $484 \pm 8$      & $4.32 \pm 0.02$   & $8.926 \pm 0.153$  & 10               \\
	$2\text{M}$       & $476 \pm 4$      & $0.72 \pm 0.02$   & $1.513 \pm 0.044$  & 10               \\
	\hline
	\end{tabular}
	\caption{Measured input voltage vs. output voltage at a range of
	frequencies, up until the slew rate is seen for the inverting amplifier.}
\end{table}

\begin{table}[H]
	\centering
	\begin{tabular}{|l|l|l|l|l|}
	\hline
	Frequency (Hz)    & $V_{\rm in}$ (mV) & $V_{\rm out}$ (V) & Measured gain (dB) & Target gain (dB) \\
	\hline
	$10$              & $500 \pm 4$      & $2.52 \pm 0.04$   & $5.04 \pm 0.09$    & 5                \\
	$1.414\text{k}$   & $484 \pm 6$      & $2.60 \pm 0.02$   & $5.372 \pm 0.078$  & 5                \\
	$53.178\text{k}$  & $472 \pm 4$      & $2.6 \pm 0.02$    & $5.508 \pm 0.063$  & 5                \\
	$326.122\text{k}$ & $476 \pm 4$      & $2.56 \pm 0.04$   & $5.378 \pm 0.095$  & 5                \\
	$807.617\text{k}$ & $480 \pm 4$      & $1.76 \pm 0.02$   & $3.667 \pm 0.052$  & 5                \\
	$2\text{M}$       & $472 \pm 4$      & $0.72 \pm 0.02$   & $1.525 \pm 0.021$  & 5                \\
	\hline
	\end{tabular}
	\caption{Measured input voltage vs. output voltage at a range of
	frequencies, up until the slew rate is seen for the inverting amplifier.}
\end{table}

\begin{table}[H]
	\centering
	\begin{tabular}{|l|l|l|l|l|}
	\hline
	Frequency (Hz)    & $V_{\rm in}$ (mV) & $V_{\rm out}$ (V) & Measured gain (dB) & Target gain (dB) \\
	\hline
	$10$              & $500 \pm 4$      & $1.6 \pm 0.02$    & $3.2 \pm 0.047$    & 3.16             \\
	$1.414\text{k}$   & $492 \pm 4$      & $1.64 \pm 0.02$   & $3.333 \pm 0.049$  & 3.16             \\
	$53.178\text{k}$  & $480 \pm 4$      & $1.6 \pm 0.02$    & $3.333 \pm 0.05$   & 3.16             \\
	$326.122\text{k}$ & $476 \pm 4$      & $1.6 \pm 0.02$    & $3.361 \pm 0.051$  & 3.16             \\
	$807.617\text{k}$ & $484 \pm 4$      & $1.64 \pm 0.02$   & $3.388 \pm 0.05$   & 3.16             \\
	$1.27\text{M}$    & $480 \pm 4$      & $1.14 \pm 0.02$   & $2.375 \pm 0.046$  & 3.16             \\
	$2\text{M}$       & $480 \pm 4$      & $0.780 \pm 0.02$  & $1.625 \pm 0.044$  & 3.16             \\
	\hline
	\end{tabular}
	\caption{Measured input voltage vs. output voltage at a range of
	frequencies, up until the slew rate is seen for the inverting amplifier.}
\end{table}

\subsection{Full oscilloscope outputs for differentiator circuit}
\begin{table}[H]
	\centering
	\begin{tabular}{|c|c|}
		\hline
		Number	& Output\\
		\hline
		1		& \includegraphics[width=0.65\textwidth]{./Figures/diff_1.jpg} \\
		\hline
		2		& \includegraphics[width=0.65\textwidth]{./Figures/diff_2.jpg} \\
		\hline
		3		& \includegraphics[width=0.65\textwidth]{./Figures/diff_3.jpg} \\
		\hline
	\end{tabular}
	\caption{
		Table of oscilloscope outputs for triangle inputs in table 
		\ref{tab:diff_inputs}.
	}
	\label{tab:diff_output_triangle}
\end{table}

\begin{table}[H]
	\centering
	\begin{tabular}{|c|c|}
		\hline
		Number	& Output\\
		\hline
		4		& \includegraphics[width=0.65\textwidth]{./Figures/diff_4.jpg} \\
		\hline
		5		& \includegraphics[width=0.65\textwidth]{./Figures/diff_5.jpg} \\
		\hline
		6		& \includegraphics[width=0.65\textwidth]{./Figures/diff_6.jpg} \\
		\hline
	\end{tabular}
	\caption{
		Table of oscilloscope outputs for square inputs in table 
		\ref{tab:diff_inputs}.
	}
	\label{tab:diff_output_square}
\end{table}

\begin{table}[H]
	\centering
	\begin{tabular}{|c|c|}
		\hline
		Number	& Output\\
		\hline
		7		& \includegraphics[width=0.65\textwidth]{./Figures/diff_7.jpg} \\
		\hline
		8		& \includegraphics[width=0.65\textwidth]{./Figures/diff_8.jpg} \\
		\hline
		9		& \includegraphics[width=0.65\textwidth]{./Figures/diff_9.jpg} \\
		\hline
	\end{tabular}
	\caption{
		Table of oscilloscope outputs for sine inputs in table 
		\ref{tab:diff_inputs}.
	}
	\label{tab:diff_output_sine}
\end{table}

\end{document}
